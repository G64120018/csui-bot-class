%%%%%%%%%%%%%%%%%%%%%%%%%%%%%%%%%%%%%%%%%%%%%%%%%%%%%%%%%%%%%%%%%%%%%%%%%%%%%%%%
%2345678901234567890123456789012345678901234567890123456789012345678901234567890
%        1         2         3         4         5         6         7         8

\documentclass[letterpaper, 10 pt, conference]{ieeeconf}  % Comment this line out if you need a4paper

%\documentclass[a4paper, 10pt, conference]{ieeeconf}      % Use this line for a4 paper

\IEEEoverridecommandlockouts                              % This command is only needed if 
                                                          % you want to use the \thanks command

\overrideIEEEmargins                                      % Needed to meet printer requirements.

% See the \addtolength command later in the file to balance the column lengths
% on the last page of the document

% The following packages can be found on http:\\www.ctan.org
\usepackage{hyperref}
\usepackage{graphicx} % for pdf, bitmapped graphics files
\graphicspath{{./fig/}}
\DeclareGraphicsExtensions{.pdf,.jpeg,.png}
\usepackage{subfigure}
\usepackage{lipsum}
\usepackage[us,12hr]{datetime} % `us' makes \today behave as usual in TeX/LaTeX
%\usepackage{graphics} % for pdf, bitmapped graphics files
%\usepackage{epsfig} % for postscript graphics files
%\usepackage{mathptmx} % assumes new font selection scheme installed
%\usepackage{times} % assumes new font selection scheme installed
%\usepackage{amsmath} % assumes amsmath package installed
%\usepackage{amssymb}  % assumes amsmath package installed

\title{\LARGE \bf
IKO42360 Technical Report: Monte Carlo Localization (MCL)*
}

\author{Vektor Dewanto$^{1}$ and Nanda Kurniawan$^{2}$ and Wisnu Jatmiko$^{3}$  % <-this % stops a space
\thanks{*Team TA, compiled on {\ddmmyyyydate\today} at \currenttime}% <-this % stops a space
\thanks{$^{1}$NPM: 0606029492,
        {\tt\small vektor.dewanto@gmail.com}}%
\thanks{$^{2}$NPM: 0606029492,
        {\tt\small vektor.dewanto@gmail.com}}%
\thanks{$^{3}$NPM: 0606029492,
        {\tt\small vektor.dewanto@gmail.com}}%
}

\begin{document}

\maketitle
\thispagestyle{empty}
\pagestyle{empty}

%%%%%%%%%%%%%%%%%%%%%%%%%%%%%%%%%%%%%%%%%%%%%%%%%%%%%%%%%%%%%%%%%%%%%%%%%%%%%%%%
\begin{abstract}
You may or may not write an abstract.
It is essentially a summary comprising of not more than 250 words.
\end{abstract}

%%%%%%%%%%%%%%%%%%%%%%%%%%%%%%%%%%%%%%%%%%%%%%%%%%%%%%%%%%%%%%%%%%%%%%%%%%%%%%%%
\section{Introduction}
This report guidance is not strict in the sense that you may \emph{not only} add some point \emph{but also} remove some point, yet you still obtain an optimal grade (if those addition and removal are reasonably justified).
Feel free to ask TAs for any doubt.

\begin{itemize}
    \item give some context or background
    \item state the problem
    \item state the objective
    \item give an overview about the method
    \item why choose a particular method
    \item give an overview about the experiment results
    \item outline the content
\end{itemize}

%%%%%%%%%%%%%%%%%%%%%%%%%%%%%%%%%%%%%%%%%%%%%%%%%%%%%%%%%%%%%%%%%%%%%%%%%%%%%%%%
\section{The Robot Simulation}
\begin{itemize}
\item robot's specification?
\item assumptions?
\end{itemize}
\subsection{Action Simulation}
\begin{itemize}
\item assumptions?
\item probability distribution?
\item code listing
\end{itemize}
\subsection{Perception Simulation}
\begin{itemize}
\item assumptions?
\item probability distribution?
\item code listing
\end{itemize}

%%%%%%%%%%%%%%%%%%%%%%%%%%%%%%%%%%%%%%%%%%%%%%%%%%%%%%%%%%%%%%%%%%%%%%%%%%%%%%%%
\section{Action Models}
\begin{itemize}
    \item the theory: either odometry or velocity motion models
    \item assumptions, e.g. gaussian errors
    \item code listing
    \item experiment on motion model
    \item reproduce fig. 5.4, 5.9, 5.10 from~\cite{Thrun:2005:PR}
\end{itemize}

%%%%%%%%%%%%%%%%%%%%%%%%%%%%%%%%%%%%%%%%%%%%%%%%%%%%%%%%%%%%%%%%%%%%%%%%%%%%%%%%
\section{Perception Models}
\begin{itemize}
    \item the theory: either beam model for range finders or feature-based measurement models
    \item assumptions, e.g. gaussian errors
    \item code listing
    \item tuning the intrinsic parameters
    \item experiment on perception model
    \item reproduce fig. 6.4, 6.5 from~\cite{Thrun:2005:PR}
\end{itemize}

%%%%%%%%%%%%%%%%%%%%%%%%%%%%%%%%%%%%%%%%%%%%%%%%%%%%%%%%%%%%%%%%%%%%%%%%%%%%%%%%
\section{KLD-sampling MCL}
\subsection{The standard MCL}
\begin{itemize}
\item the theory
\item code listing
\end{itemize}

\subsection{The KLD-sampling MCL}
You may try other variants of MCL.
If you do so, do not forget to change the title of this subsection and the section.
\begin{itemize}
\item the theory
\item code listing
\end{itemize}

%%%%%%%%%%%%%%%%%%%%%%%%%%%%%%%%%%%%%%%%%%%%%%%%%%%%%%%%%%%%%%%%%%%%%%%%%%%%%%%%
\section{Experiments and Results}

\subsection{Setup}
\begin{itemize}
    \item what kind of map? assumptions?
    \item code listing for map constructions
    \item draw/illustrate the map!
    \item control commands? hardcoded? wall following?
    \item experiment design or procedure; the start state?
    \item evaluation metrics: errors, convergence time
    \item scope: localization problem type?
    \item is a live simulation available? excellent if one exists
    \item for each experiment (local, global and kidnapped-robot), reproduce fig. 8.3, 8.4, 8.7, 8.11, 8.13, 8.16, 8.17, 8.18, 8.19 from~\cite{Thrun:2005:PR}. For some figures, replace ``cell size'' with ``number of particles''
\end{itemize}

\subsection{Local Localization}
\begin{itemize}
\item result, comparison: standard vs. variant
\item analysis
\end{itemize}

\subsection{Global Localization}
\begin{itemize}
\item result, comparison: standard vs. variant
\item analysis
\end{itemize}

\subsection{Kidnapped-robot Problem}
\begin{itemize}
\item result, comparison: standard vs. variant
\item analysis
\end{itemize}

%%%%%%%%%%%%%%%%%%%%%%%%%%%%%%%%%%%%%%%%%%%%%%%%%%%%%%%%%%%%%%%%%%%%%%%%%%%%%%%%
\section{Related Work}
Discuss or review in-depth \emph{one} related work.
Each team is initially assigned different paper.
This can be changed to any paper of your choice; preferably it is published in major robotics conferences, such as ICRA, IROS, RSS.
For you, we have collected: \cite{1389783, 5354298, 5509950, 6696380, 6094843, 4059089}
\begin{itemize}
    \item what is the problem that they tackled?
    \item how is the proposed solution/method?
    \item how good is their solution? any comparisons?
    \item what it the limitations of their work?
    \item what future work or open problems do they mention?
    \item your opinions about the work
\end{itemize}
%%%%%%%%%%%%%%%%%%%%%%%%%%%%%%%%%%%%%%%%%%%%%%%%%%%%%%%%%%%%%%%%%%%%%%%%%%%%%%%%
\section{Conclusions}
\begin{itemize}
    \item re-state the problem, goal briefly
    \item highlight the (proposed) method
    \item convey the results, mention the advantages of using the (proposed) method and current limitations
    \item some lessons-learned, some open problems
\end{itemize}

%%%%%%%%%%%%%%%%%%%%%%%%%%%%%%%%%%%%%%%%%%%%%%%%%%%%%%%%%%%%%%%%%%%%%%%%%%%%%%%%
\section*{APPENDIX}
The answer to Problem 2 is explained here.
\begin{itemize}
    \item re-state the problem, goal briefly
    \item formulate the $MDP=(S, A, P_{sa}, R, \gamma)$
    \item code listing
    \item plot the values of $V^*$ on the grid map
    \item plot the optimal policy~$\pi^*$ on the grid map
    \item analysis
\end{itemize}

%%%%%%%%%%%%%%%%%%%%%%%%%%%%%%%%%%%%%%%%%%%%%%%%%%%%%%%%%%%%%%%%%%%%%%%%%%%%%%%%
\bibliographystyle{IEEEtran}
\bibliography{IEEEabrv,rpt}

\end{document}
