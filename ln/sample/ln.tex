\documentclass{article}
\usepackage{iko42360}

%% the format for the lecture environment is
% \begin{lecture}{lecture number}{lecture title}{scribe(s)}{lecture date}
\begin{lecture}{10}{Probabilistics Robotics: Introduction}{Your Name}{04/28/2014}

\section{Introduction}
Some story about robotics today.
For example, the Stanford's tidy-up robot project.
Some story about robotics today.
For example, the Stanford's tidy-up robot project.

Robots' basic capabilities.
What makes a robot?
Why coding a robot is so hard?
What do we do in AI for robotics?
Robots' basic capabilities.
What makes a robot?
Why coding a robot is so hard?
What do we do in AI for robotics?

Why use probabilistics~\cite{Thrun:2005:PR}?
Why bother with uncertainty?
Where does the uncertainty come from?
Why use probabilistics?
Why bother with uncertainty?
Where does the uncertainty come from?

The theorem of total probability?
This is text style:
$\lim_{n \to \infty}
\sum_{k=1}^n \frac{1}{k^2}
= \frac{\pi^2}{6}$.
And this is display style:
\begin{equation}
\lim_{n \to \infty}
\sum_{k=1}^n \frac{1}{k^2}
= \frac{\pi^2}{6}
\end{equation}

\section{The Markov Assumption}
Illustrate the Markov Decision Processes?
Why is a state said to have the Markov property?
Explain the Markov assumption?
Illustrate the Markov Decision Processes?
Why is a state said to have the Markov property?
Explain the Markov assumption?

\begin{equation*}
\mathbf{X} = \left(
\begin{array}{ccc}
x_1 & x_2 & \ldots \\
x_3 & x_4 & \ldots \\
\vdots & \vdots & \ddots
\end{array} \right)
\end{equation*}

\section{Action Models}
What is the transition probability?
Explain some example?
Illustrate the belief of a mobile robot about its position that is inferred merely from its odometry using encoders!
What is the transition probability?
Explain some example?
Illustrate the belief of a mobile robot about its position that is inferred merely from its odometry using encoders!

\begin{equation*}
|x| = \left\{
\begin{array}{rl}
-x & \text{if } x < 0,\\
0 & \text{if } x = 0,\\
x & \text{if } x > 0.
\end{array} \right.
\end{equation*}

\section{Perception Models}
How to obtain a measurement probability?
What is the implication of the independent noise assumption?
\begin{equation*}
\sum_{i=1}^n \qquad
\int_0^{\frac{\pi}{2}} \qquad
\prod_\epsilon
\end{equation*}

\section{The Bayes Filter}
\begin{theorem}\cite[Theorem 1]{Thrun:2005:PR}: For all $\mathbf{x}\in \mathcal{A}^\infty$, the sequence of denoisers $\{\hat{X}^n_{\mathrm{univ}}\}$ defined in (14) satisfies
a) $\lim_{n\rightarrow\infty}\big[L_{\hat{X}^n_{\mathrm{univ}}}(x^n,Z^n)-D_{k_n}(x^n,Z^n)\big]=0$, a.s.
b) $E\big[L_{\hat{X}^n_{\mathrm{univ}}}(x^n,Z^n)-D_{k_n}(x^n,Z^n)\big]=O\big(\sqrt{\frac{k_n M^{2k_n}}{n}}\big)$
\end{theorem}

\section{Miscellany}
This was prepared by Vektor, who gratefully acknowledges Dr.~Eng.~Wisnu Jatmiko for such teaching opportunity.
You may contact him at vektor.dewanto@gmail.com.

\bibliographystyle{IEEEtran}
\bibliography{IEEEabrv,ref}

\end{lecture}
\theend
